\documentclass{article}

\usepackage{polski}
\usepackage{hyperref}
\usepackage{amssymb}
\usepackage{gensymb}

\title{Room heating simulation}
\author{\textbf{Damian Tworek, Kamil Miśkowiec} \\ \textbf{Jan Ziętek, Łukasz Wala}\\
    \textit{AGH, Wydział Informatyki, Elektroniki i Telekomunikacji} \\
    \textit{Modelowanie Systemów Dyskretnych 2021/2022}}
\date{Kraków, \today}

\begin{document}
\maketitle

\section{Wstęp}
Ten projekt prezentuje metodę modelowania procesu rozprzestrzeniania się ciepła w pomieszczeniu z użyciem
trójwymiarowego automatu komórkowego, który efektywnym, jednak uproszczonym rozwiązaniem. 
Model bazowany jest na dwóch zjawiskach: przewodnictwie cieplnym oraz konwekcji.
Uwzględnia on również wpływ materiałów, ze których zbudowane są ściany, sufit itp., czynników takich jak temperatura na
zewnątrz.

\section{Mechanizmy wymiany ciepła}
\subsection{Przewodzenie cieplne}
Energia przepływająca w skutek przewodnictwa cieplnego może być opisana prawem Fouriera. Jego postać różniczkowa to
$$q=-k\triangledown T$$
gdzie:
\begin{itemize}
    \item 
    $q$ --- gęstość strumienia ciepła (ang. \textit{heat flux density}, $\frac{W}{m^2}$),
    \item
    $k$ --- przewodność cieplna ($\frac{W}{m\cdot K}$),
    \item
    $\triangledown T$ --- gradient temperatury ($\frac{K}{m}$).
\end{itemize}

\subsection{Konwekcja}
Konwekcja jest procesem przepływu ciepła w skutek poruszania się cieczy lub gazu. Może być opisana przy pomocy
prawa stygnięcia (Newtona)
$$\frac{Q}{dt}=hA(T_o - T_s)$$
gdzie:
\begin{itemize}
    \item 
    $Q$ --- przekazywane ciepło ($J$),
    \item
    $dt$ --- przyrost czasu ($s$),
    \item
    $h$ --- współczynnik konwekcji ($\frac{W}{m^2 \cdot K}$),
    \item
    $A$ --- powierzchnia ($m^2$),
    \item
    $T_o - T_s$ --- różnica temperatur pomiędzy obiektem oraz cieczą/gazem ($K$).
\end{itemize}

\section{Model}
\subsection{Sąsiedztwo}
Zastosowanym modelem sąsiedztwa dla trójwymiarowej tablicy reprezentującej pomieszcenie będzie sąsiedztwo z komórkami
wspódzielącymi ściany, to znaczy że zbiór sąsiadów $S$ komórki $k$ o współrzędnych $a,b,c$ jest następujący
$$S_{a,b,c}=\{k_{a-1,b,c}, k_{a+1,b,c}, k_{a,b-1,c}, k_{a,b+1,c}, k_{a,b,c-1}, k_{a,b,c+1}, \}$$
z wyjątkiem komórek na granicach tablicy, których zbiory sąsiadów są odpowiednio pomniejszone o nieistniejące komórki.

\subsection{Przewodzenie cieplne}
Zgodnie z prawem Fouriera dla przypadku jednowymiarowego, wówczas z uwzględnieniem wszystkich sąsiadów
wzór na przekazane ciepło
\begin{equation}
    \displaystyle Q=-\sum_{j \in S(i)}k\frac{\Delta T_j}{\Delta l}A
    \end{equation}
gdzie,\\
$S(i)$ --- zbiór sąsiadó komórki $i$, \\
$Q$ --- przepływ ciepła ($W$), \\
$k$ --- przewodnictwo cieplne ($\frac{W}{m \cdot K}$), \\
$\Delta T_i$ --- różnica temperatur pomiędzy komórką $i$ a jej sąsiadem $j$ (K), \\
$l$ --- odległość środkami komórek ($m$), \\
$A$ --- pole powierzchni wymiany ($m^2$). \\

Następnie temperaturę komórki można obliczyć ze wzoru

\begin{equation}
T=\frac{Q}{\rho V c}
\end{equation}

gdzie, \\
$Q$ --- przepływ ciepła ($W$), \\
$\rho$ --- gęstość ($\frac{kg}{m^3}$), \\
$V$ --- objętość ($m^3$), \\
$c$ --- ciepło właściwe ($\frac{K}{Kg\cdot K}$)

\subsection{Konwekcja}
Przy konwekcji uwzględniane są tylko wartości górnego oraz dolnego sąsiada komórki $i$, wówczas zależność opisana jest wzorem
\begin{equation}
Q= -hA dt ((T_i - T_g) + (T_i - T_d))
\end{equation}
gdzie,
$Q$ --- przepływ ciepła ($W$), \\
$h$ --- współczynnik konwekcji, \\
$A$ --- powierzchnia styku komórek ($m^2$), \\
$dt$ --- przyrost czasu ($s$), \\
$T_i$, $T_g$, $T_d$ --- temperatury kolejno badanej komórki, górnego i dolnego sąsiada ($K$).

Przepływ ciepła występuje jednak tylko przy założeniu, że temperatura komórki dolnej jest większa niż komórki badanej (analogicznie
temperatura komórki badanej jest większa niż komórki górnej). Również w momencie, kiedy ciepło dotrze pod sufit, tzn. nie ma nad
nim więcej komórek z powietrzem, zaczyna się rozchodzić na boki, co ma symulować wypychanie powietrza zimnego z pod sufitu. 
Następnie można użyć wzoru \textbf{2} do obliczenia temperatury.

\subsection{Przepływ ciepła przez ściany, okna itp.}
Powyższe metody opisują zachowanie w obszarze komórek o typie powietrza, natomiast symulacja uwzględnia również:
\begin{itemize}
    \item 
    ściany, sufity, podłogi, okna o różnych właściwościach termoprzewodzących,
    \item
    źródło ciepła, np. grzejnik.
\end{itemize}

Temperatura tych obszarów nie jest wyliczana, jednak mają one istotny wpływ na ilość ciepła uciekającego z pomieszczenia, jak
i pojawiającego się w nim. Oczywistym podejściem byłoby zamodelowanie np. ścian o adekwatnej grubości (składających się 
z pewnej liczby warstw komórek) i modelowania procesów wewnątrz nich. Innym, zbadanym i prostszym rozwiązaniem jest wykorzystanie
zjawiska przepuszczalności cieplnej (ang. \textit{thermal transmittance}), dzięki której do zamodelowania ściany wystarczy
jedna warstwa komórek, która symuluje ścianę wielowarstwową dzięki wykorzystaniu odpowiedniego współczynnika \textit{u-value}.

\begin{equation}
Q = -AU(T_p-T_m)dt
\end{equation}
gdzie,
$Q$ --- przepływ ciepła ($W$), \\
$A$ --- powierzchnia styku komórek ($m^2$), \\
$U$ --- \textit{u-value}, \\
$dt$ --- przyrost czasu ($s$), \\
$T_p$, $T_m$ --- temperatury kolejno badanej komórki z powietrzem, sąsiada o innym typie ($K$).

Analogicznie, do obliczenia temperatury można użyć wzoru \textbf{2}.

\section{Walidacja modelu}
Pierwszym, najmniej formalnym krokiem walidacji, było przetestowanie pewnych scenariuszy oraz ich skutków, np.:
\begin{itemize}
    \item
    Przy stałym ogrzewaniu, współczynnikach przenikalności cieplnej ścian i temperaturze zewnątrznej, po pewnym czasie temepratura
    w pomieszczeniu stablizuje się, a jej jest zgodna z intuicją (np. ok 20$\degree$C przy źródle ciepła o temperaturze 60$\degree$C i 
    temperaturze na zewnątrz rzędu kilku-kilkunastu $\degree$C).
    \item 
    Porównanie prób, gdzie w jednej temperatura na zewnątrz wynosiła 15$\degree$C z próbą, gdzie wynosiła 0$\degree$C. Po pewnym czasie,
    gdy temperatura w pomieszczeniu przy stałym ogrzewaniu stablizuje się, w próbie z mniejszą temperaturą na zewnątrz jest ona mniejsza.
    \item
    Zastosowanie nierealistycznie niskich wartości przenikalności cieplnej ścian, podłogi itd. skutkuje bardzo wyskoą temperaturą
    wewnątrz pomieszczenia, ponieważ, zgodnie przypuszczeniami, ciepło jest w układzie bardzo dobrze izolowanym i nie może z niego uciec.
    \item
    Zastosowanie podobnego scenariusza jak powyższy, z tą różnicą, że wspólczynniki prznikalności okien są na realistycznym poziomie
    skutowało obszarem chłodniejszego powietrza w okolicy okna. 

\end{itemize}


\section{Wnioski}
Celem projektu było stworzenie modelu, który symulowałby rozprzesztrzenianie się ciepła w pomieszczeniu za pomocą automatu komórkowego.
Stworzony model uwzględnia zjawiska przewodnictwa cieplnego wewnątrz powietrza oraz na granicach powietrze-materiał, 
co pozwala uwzględnić stopień izolacji cieplnej pomieszczenia. Model uwzględnia również proces konwekcji. Wykorzystanie automatu komórkowego
zapewnia bardzo dobrą efektywność symulacji.

Cel ten został osiągnięty w pewnym stopniu, model posiada kilka niedoskonałości, np. brak promieniowania cieplnego
oraz niedokładne odtworzenie procesu konwekcji, jednak mimo tego pozwala z pewnym przybliżeniem przewidzieć symulowany proces, wyniki symulacji
nie odbiegają bardzo od danych porównawczych, co oznacza, że przy dalszym rozwoju, model mógłby prezentować dobrą skuteczność.



\section{Źródła}
\begin{enumerate}
    \item
    Jarosław Wąs, Artur Karp, Szymon Łukasik, Dariusz Pałka: \textit{Modeling of Fire Spread Including Different 
    Heat Transfer Machanisms Using Cellular Automata},
    \item
    Purvesh Bharadawaj, Ljubomir Jankovic: \textit{Cellular Automata Simulation of Three-dimensional Building Heat Loss},
    \item
    Philip Kosky, Robert Balmer, William Keat, George Wise: \textit{Exploring Engineering},
\end{enumerate}

\end{document}