\documentclass{article}

\usepackage{polski}
\usepackage{hyperref}

\title{Room heating simulation - harmonogram}
\author{\textbf{Damian Tworek, Kamil Miśkowiec} \\ \textbf{Jan Ziętek, Łukasz Wala}\\
    \textit{AGH, Wydział Informatyki, Elektroniki i Telekomunikacji} \\
    \textit{Modelowanie Systemów Dyskretnych 2021/2022}}
\date{Kraków, \today}

\begin{document}
\maketitle

\section{Idea projektu}
Projekt polega na stworzeniu symulacji ogrzewania pomieszczenia. Na podstawie symulacji wyciągnięte zostaną
wnioski dotycznące najbardziej opłacalnej strategii ogrzewania pomieszczeń. 

Podstawą symulacji będzie trójwymiarowy automat komórkowy działający na podstawie wybranego modelu dynamiki termicznej,
opierający się przede wszystkim na przewodnictwie cieplnym (możliwość uwzględnienie np. konwekcji czy promieniowania, wilgotność powietrza,
przepływ powietrza raczej nie zostanie uwzględniony ze względu na poziom skomplikowania).
Będzie on również uwzględniał takie elementy jak: przenikalność cieplna elementów takich jak ściany, podłoga, sufit, okna, do tego temperaturę
otoczenia poza pomiesczeniem (na zewnątrz, u sąsiadów, jeżeli to np. pomieszczenie w bloku), źródła ciepła: grzejniki, ludzi znajdujących
się w pomieszczeniu.

\section{Harmonogram}
\subsection{Zajęcia 04.05.2022}

\begin{itemize}
    \item
    Wybór technologii użytej do wizualizacji automatu komórkowego.
    \item
    Podstawowa implementacja warstwy wizualizacyjnej.
    \item
    Specyfikacja użytego modelu uproszczonej dynamiki termicznej.
\end{itemize}

\subsection{Zajęcia 11.05.2022}

\begin{itemize}
    \item
    Implementacja używanego modelu dynamiki termicznej.
    \item
    Początek implementacji obiektów bedących elementami symulowanego pomieszczenia, np. ściany, grzejnik, okna.
\end{itemize}

\subsection{Zajęcia 18.05.2022}

\begin{itemize}
    \item
    Dokończenie implementacji elementów pomieszcznia.
    \item
    Implementacja interaktywnej możliwość edycji parametrów symulacji.
\end{itemize}

\subsection{Zajęcia 25.05.2022}

\begin{itemize}
    \item
    Poprawki oraz ostateczne usprawnienie symulacji oraz jej wizualizacji.
    \item
    Implementacja mechnizmów kolekcjonujących dane z symulacji.
    \item
    Wykonanie symulacji z różnymi paramatrami oraz zebranie danych.
\end{itemize}

\subsection{Zajęcia 01.06.2022}

\begin{itemize}
    \item
    Analiza wyników, weryfikacja oraz sformuowanie wniosków.
    \item
    Oddanie projektu oraz przedstawienie uzyskanych wniosków.
\end{itemize}

\section{Literatura}

\begin{enumerate}
    \item
    Jarosław Wąs, Artur Karp, Szymon Łukasik, Dariusz Pałka \textit{Modeling of Fire Spread Including Different 
    Heat Transfer Machanisms Using Cellular Automata}

    Ta praca traktuje o rozprzestrzenianiu się ognia, jednak można z niej zaczerpnąć kilka wartościowych informacji
    istotnych dla projektu, jak rodzaj sąsiedztwa dla automatu trójwymiarowego dla sześciennych komórek. 
    W projekcie zostanie zastosowane sąsiedztwo zawierające 6 komórek przylegających ścianami. Inne, odrzucone możliwości
    to komórki współdzielące krawędzie lub komórki współdzielące wierzchołki, jednak z powodów opisanych w pracy zostały odrzucone. 
    Do tego praca specyfikuje wzory opisujące zjawisko konwekcji.

    \item
    Purvesh Bharadawaj, Ljubomir Jankovic \textit{Cellular Automata Simulation of Three-dimensional Building Heat Loss}

    Z tej pracy zaczerpnięty został podział typów agentów odpowiadający elementom o różnych własnościach termodynamicznych 
    (powietrze, ściany, okna, izolacja, źródła ciepła).

    Również na podstawie tej oraz poprzedniej pracy stworzony został model przewodnictwa cieplnego oparty na prawie Fouriera,
    przystosowany do użycia w automacie komórkowym:

    Dla pewnej komórki $i$ przepływ ciepła można opisać równaniem
    \begin{equation}
    \displaystyle Q=-\sum_{j \in O(i)}k\frac{\Delta T_j}{\Delta l}A
    \end{equation}

    Gdzie,\\
    $O(i)$ - zbiór sąsiadó komórki $i$, \\
    $Q$ - przepływ ciepła (W), \\
    $k$ - przewodnictwo cieplne (W/(m$\cdot$K)), \\
    $\Delta T_i$ - różnica temperatur pomiędzy komórką $i$ a jej sąsiadem $j$ (K), \\
    $l$ - odległość środkami komórek (m), \\
    $A$ - pole powierzchni wymiany (m$^2$). \\
    
    Następnie temperaturę komórki można obliczyć ze wzrou

    \begin{equation}
    T=\frac{Q}{\rho V c}
    \end{equation}

    Gdzie, \\
    $Q$ - przepływ ciepła (W), \\
    $\rho$ - gęstość (kg/m$^3$), \\
    $V$ - objętość (m$^3$), \\
    $c$ - ciepło właściwe (K/(Kg$\cdot$K))

    \item
    Sergey Bobkov, Edward Galiaskarov, Irina Astrakhantseva \textit{The use of cellular automata systems for
    simulation of transfer processes in a non-uniform area}

    \item
    Sergey Bobkov \textit{Cellular Automata Systems application for simulation of some processes in solids}

    Dwie ostatnie pozycje zawierają alternatywne podejście do problemu, które może być przydatne podczas walidacji modelu lub
    jeżeli w trakcie tworzenia okaże się, że wybrany model jest w jakiś sposób niepoprawny.
\end{enumerate}


\end{document}  