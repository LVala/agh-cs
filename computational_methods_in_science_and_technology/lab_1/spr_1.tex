\documentclass{article}

\usepackage{polski}
\usepackage{amsmath}
\usepackage{graphicx}
\usepackage{float}

\title{Prezentacja i opracowanie: wielomian Wilkinsona}
\author{\textbf{Łukasz Wala}\\
    \textit{AGH, Wydział Informatyki, Elektroniki i Telekomunikacji} \\
    \textit{Metody Obliczeniowe w Nauce i Technice 2021/2022}}
\date{Kraków, \today}

\begin{document}
\maketitle

\section{Treść zadania}
Dany jest wielomian Wilkinsona $W(x)=(x-1)(x-2)(x-3)...(x-20)$.
\begin{itemize}
    \item
    Przekształcić wielomian do postaci $W(x)=x^{20}+a_{19}x^{19}+...+a_1x+a_0$. Podać wyznaczone wartości współczynników $a_i$.
    \item
    Dla $x=1$ oraz $x=20$ dokładna wartość wielomianu wynosi $0$. Obliczyć warość wielomianu dla $x=1$ oraz $x=20$. Do wyznaczenia wartości wielomianu wykorzystać schemat Hornera oraz dowolny schemat sumacyjny.
\end{itemize}

\section{Przekształcenie wielomianu}
\textbf{Uwaga!} Kod programu w języku C użytego do rozwiązania zadania znajduje w pliku \verb|main.c|.
Program można uruchomić przez wywołanie w terminalu komendy \verb|make run|.

Do przekształcenia wielomianu użyty został naiwny algorytm mnożenia wielomianów wymnażający jego kolejne czynniki.
Z racji, że wykonywane operacje to dodawanie i dzielenie liczb całkowitch, typ danych użyty do przchowywania współczynników $a_i$
może być zarówno stałoprzecinkowy lub zmiennoprzecinkowy, o ile pomieści owe współczynniki, np \verb|long long int| czy \verb|double|.
Współczynniki obliczone przez załączony program są następujące
\begin{verbatim}
a0: 2432902008176640000
a1: -8752948036761600000
a2: 13803759753640704000
a3: -12870931245150988800
a4: 8037811822645051776
a5: -3599979517947607200
a6: 1206647803780373360
a7: -311333643161390640
a8: 63030812099294896
a9: -10142299865511450
a10: 1307535010540395
a11: -135585182899530
a12: 11310276995381
a13: -756111184500
a14: 40171771630
a15: -1672280820
a16: 53327946
a17: -1256850
a18: 20615
a19: -210
a20: 1
\end{verbatim}
Zgadzają się one z wartościami rzeczywistymi.

\section{Obliczenie wartości wielomianu}
Wartości wielomianu dla $x=1$ oraz $x=20$ zostały obliczone na dwa sposoby:
\begin{itemize}
    \item
    z użyciem naiwnego schematu sumacyjnego
    \item
    z użyciem schematu Hornera
\end{itemize}
Tutaj obliczenia przeprowadzone zostały dla trzech różnych typów:
\verb|double|, \verb|long double| oraz \verb|long long|
\begin{verbatim}
-----DOUBLE-----
WYNIK x=1.0 HORNER: 1024.000000000000000
WYNIK x=1.0 SUMA: 1184.000000000000000
WYNIK x=20.0 HORNER: -27029504000.000000000000000
WYNIK x=20.0 SUMA: 3848290697216.000000000000000

-----LONG DOUBLE-----
WYNIK x=1.0 HORNER: 0.000000000000000
WYNIK x=1.0 SUMA: 0.000000000000000
WYNIK x=20.0 HORNER: 0.000000000000000
WYNIK x=20.0 SUMA: 2147483648.000000000000000

-----LONG LONG INT-----
WYNIK x=1.0 HORNER: 0
WYNIK x=1.0 SUMA: 1739658311799032064
WYNIK x=20.0 HORNER: 0
WYNIK x=20.0 SUMA: -9223372036854775808
\end{verbatim}

Wyniki dla \verb|double| bardzo ewidentnie się nie zgadzają z rzeczywistością, ponieważ, mimo że ten typ
pozwalał na poprawne przedstawienie współczynników, chwilowe wartości otrzymywane podczas
obliczeń znacznie wykraczają poza zakres, gdzie wszystkie liczby całkowite są bezbłędnie reprezentowane.

W przypadku \verb|long long| i \verb|long double| dla schematu Hornera, gdzie chwilowe wartości
podczas wykonywania obliczeń są mniejsze, wynik jest poprawny. Dla naiwnego sumowania, niektóre wartości
prawdopodobnie nadal wkraczają w zakres, gdzie liczby całkowite nie są reprezentowane dokładnie (dla \verb|long double|)
lub overflow'ują (dla \verb|long long|), stąd niepoprawne wyniki.

Inną ciekawą obserwacją jest to, jak bardzo na wynik wpływają zmiany współczynników. Przy zmianie $a_{19}$ o wartość $0.000000000000001$
wyglądają następująco
\begin{verbatim}
-----LONG DOUBLE-----
WYNIK x=1.0 SHIFTED HORNER: 0.000000000000000
WYNIK x=20.0 SHIFTED HORNER: 1200000000.000000000000000
\end{verbatim}
Dla $x=1$ różnica jest niezauważalna dla 17 miejsc dziesiętnych, natomiast dla $x=20$ jest już bardzo duża.

\end{document}