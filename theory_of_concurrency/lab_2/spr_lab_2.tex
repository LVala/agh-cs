\documentclass{article}

\usepackage{polski}
\usepackage{amsmath, array}
\usepackage{graphicx}
\usepackage{float}
\usepackage{subfig}
\usepackage{multirow}

\title{Laboratorium 2}
\author{\textbf{Łukasz Wala}\\
    \textit{AGH, Wydział Informatyki, Elektroniki i Telekomunikacji} \\
    \textit{Teoria Współbieżności 2022/23}}
\date{Kraków, \today}

\begin{document}
\maketitle

\section{Treść zadania}
\begin{enumerate}
    \item
    Zaimplementować semafor binarny za pomocą metod \textit{wait} i \textit{notify}, użyć go do synchronizacji programu \textit{Wyścig}.
    \item
    Pokazać, że do implementacji semafora za pomocą metod \textit{wait} i \textit{notify} nie wystarczy instrukcja \textit{if}, tylko potrzeba użyc \textit{while}. 
    Wyjasnić teoretycznie dlaczego i potwierdzic eksperymentem w praktyce. (wskazówka: rozważyc dwie kolejki: 
    czekajaca na wejście do monitora obiektu oraz kolejkę zwiazaną z instrukcją \textit{wait}, rozważyc kto i kiedy jest budzony i kiedy nastepuje wyścig).
    \item
    Zaimplementować semafor licznikowy (ogólny) za pomocą semaforów binarnych. Czy semafor binarny jest szczególnym przypadkiem semafora ogólnego?
\end{enumerate}

\section{Semafor binarny}

Pierwszym krokiem rozwiązania jest zaimplementowanie semafora binarnego:

\begin{verbatim}
class BinarySemaphore {
    private boolean state;

    public BinarySemaphore() {
        this.state = true;
    }

    public synchronized void P() {
        while (!state) {
            try {
                wait();  
            }
            catch (InterruptedException e) {
                System.exit(0);
            }
        }

        state = false;
    }
    
    public synchronized void V() {
        state = true;
        notifyAll();
    }
}                
\end{verbatim}

Stworzony semafor może zostać użyty do zsynchronizowania zadania z poprzedniego laboratorium:

\begin{verbatim}
class Counter {
    private int _val;
    public Counter(int n) {
        _val = n;
    }
    public void inc() {
        _val++;
    }
    public void dec() {
        _val--;
    }
    public int value() {
        return _val;
    }
}

class IThread extends Thread {
    private Counter counter;
    private BinarySemaphore semaphore;

    public IThread(Counter counter, BinarySemaphore semaphore) {
        this.counter = counter;
        this.semaphore = semaphore;
    }

    public void run() {
        for (int i=0; i<10_000; ++i) {
            semaphore.P();
            counter.inc();
            semaphore.V();
        }
    }
}

class DThread extends Thread {
    private Counter counter;
    private BinarySemaphore semaphore;

    public DThread(Counter counter, BinarySemaphore semaphore) {
        this.counter = counter;
        this.semaphore = semaphore;
    }

    public void run() {
        for (int i=0; i<10_000; ++i) {
            semaphore.P();
            counter.dec();
            semaphore.V();
        }
    }
}

public class Race {
    public static void main(String[] args) throws InterruptedException {

        Counter cnt = new Counter(0);
        BinarySemaphore semaphore = new BinarySemaphore();
        DThread dthread = new DThread(cnt, semaphore);
        IThread ithread = new IThread(cnt, semaphore);

        dthread.start();
        ithread.start();
        dthread.join();
        ithread.join();

        System.out.println(cnt.value());
    }
}    
\end{verbatim}

Tym sposobem możemy skutecznie zaimplementować licznik.

\section{Implementacja semafora - dlaczego \textit{while}?}

Mogłoby się wydawać, że w powyższej implementacji semafora możnaby zastąpić pętlę \textit{while} 
wyrażeniem \textit{if}. Niestety jednak w przypadku, gdy zostanie użyty \textit{if}, implementacja 
nie będzie poprawnie działać z kilku powodów:

\begin{enumerate}
    \item
    wątek może wybudzić się nie będąc wywołanym przez funkcję \textit{notify}, przerwanym itp. (ang. \textit{spurious wakeup}).
    Co prawda jest to bardzo żadkie, jednek użycie pętli \textit{while} zapewnia, że jeżeli warunek
    nie będzie spełniony, wątek nie zostanie wybudzony.
    \item
    jeżeli do wybudzenia wątków użyta jest funkcja \textit{notifyAll}, wówczas wszystkie wątki zablokowane
    przez semafor zostaną odblokowane i zaczną konkurować o to, który wznowi wykonywanie jako pierwszy.
    Dzięki użyciu pętli \textit{while} wszystkie wątki, oprócz pierwszego, który się wykona, wejdą w
    stan oczekiwania, co jest pożądanym zachowaniem. Użycie wyrażenia \textit{if} poskutkowałoby odblokowaniem
    wszystkich wątków.
\end{enumerate}

\begin{verbatim}
...
public synchronized void P() {
    if (!state) {  // if instead of while
        try {
            wait();  
        }
        catch (InterruptedException e) {
            System.exit(0);
        }
    }

    state = false;
}
...
\end{verbatim}

Powyższ implementacja nie zapewnia poprawnej synchronizacji.

\section{Semafor licznikowy}

Za pomocą semafora binarnego można zaimplementować semafor licznikowy:

\begin{verbatim}
class CountingSemaphore {
    private int count;
    private BinarySemaphore binarySemaphore;

    public CountingSemaphore() {
        this.count = 1;
        this.binarySemaphore = new BinarySemaphore();
    }

    public void P() {
        binarySemaphore.P();
        synchronized (this) {
            --count;
            if (count > 0) {
                binarySemaphore.V();
            }
        }
    }
    
    public synchronized void V() {
        ++count;
        if (count == 1) {
            binarySemaphore.V();
        }
    }
}        
\end{verbatim}

Semafor licznikowy jest typem semafora, który pozwala na jednoczesny dostęp $N$ wątków do
współdzielonego stanu/sekcji krytycznej. 
Jak można zauważyć, semafor binarmy jest semaforem licznikowym dla $N=1$.

\section{Wnioski}

Semafory są prostym, ale skutecznym sposobem na ograniczenie dostępu wielu równolegle działających
wątków do współdzielonego stanu i uniknięcia problemów z dostępem do sektcji krytycznej/atomicznością operacji.
Mogą zostać zaimplementowane za pomocą intrucji \textit{wait} oraz \textit{notify} w języku Java.
Semafor licznikowy pozwala na dostęp $N$ wątków do współdzielonej sekcji, nastomiast semafor binarny pozwala
na dostęp tylko jednemu wątkowi.

\section{Bibliografia}

\begin{enumerate}
    \item 
    Dokumentacja języka Java - docs.oracle.com

\end{enumerate}


\end{document}